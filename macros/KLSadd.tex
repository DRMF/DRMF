\documentclass[twoside,11pt]{article}
\setlength{\textwidth}{16cm}
\setlength{\textheight}{22cm}
\setlength{\topmargin}{0cm}
\setlength{\oddsidemargin}{0.2mm}
\setlength{\evensidemargin}{0.2mm}
\usepackage{amsmath}
\usepackage{amssymb}
\usepackage{amsthm}
\usepackage{url}
\newcommand\sa{\smallskipamount}
\newcommand\sLP{\\[\sa]}
\newcommand\sPP{\\[\sa]\indent}
\newcommand\CC{\mathbb{C}}
\newcommand\ZZ{\mathbb{Z}}
\newcommand\al\alpha
\newcommand\be\beta
\newcommand\ga\gamma
\newcommand\de\delta
\newcommand\tha\theta
\newcommand\la\lambda
\newcommand\om\omega
\newcommand\Ga{\Gamma}
\newcommand\half{\frac12}
\newcommand\thalf{\tfrac12}
\newcommand\iy\infty
\newcommand\wt{\widetilde}
\newcommand\Zpos{\ZZ_{>0}}
\newcommand\Znonneg{\ZZ_{\ge0}}
\newcommand\const{{\rm const.}\,}
\newcommand{\hyp}[5]{\,\mbox{}_{#1}F_{#2}\!\left(
  \genfrac{}{}{0pt}{}{#3}{#4};#5\right)}
\newcommand{\qhyp}[5]{\,\mbox{}_{#1}\phi_{#2}\!\left(
  \genfrac{}{}{0pt}{}{#3}{#4};#5\right)}
\newcommand\LHS{left-hand side}
\newcommand\RHS{right-hand side}

\begin{document}

\title{Additions to Chapters 9 and 14 in the book
``Hypergeometric orthogonal polynomials and their $q$-analogues''
by Koekoek, Lesky \& Swarttouw}
\author{Tom H. Koornwinder,
\small\tt T.H.Koornwinder@uva.nl}

\date{May 3, 2013}
\maketitle
%until 93
\subsection*{Introduction}
This informal manuscript contains some formulas about ($q$)-hypergeometric
orthogonal polynomials which I missed but wanted to use
while consulting Chapters 9 and 14 in the book
\sLP
R. Koekoek, P.~A. Lesky and R.~F. Swarttouw,
{\em Hypergeometric orthogonal polynomials and their $q$-analogues},
Springer-Verlag, 2010.
\sLP
These chapters form together the (slightly extended) successor of the report
\sLP
R.~Koekoek and  R.~F. Swarttouw,
{\em The Askey-scheme of hypergeometric orthogonal
polynomials and its $q$-analogue},
Report 98-17, Faculty of Technical Mathematics and Informatics,
Delft University of Technology, 1998;
\url{http://aw.twi.tudelft.nl/~koekoek/askey/}.
\sPP
Usually, any type of formula I give for a special class of polynomials, will suggest
a similar formula for many other classes, but I have not aimed at completeness
by filling in a formula of such type at all places. The resulting choice of formulas is
rather arbitrary, just depending on the formulas which I happened to need or which raised my interest.
For each formula I give  a suitable reference or I sketch a
proof.
It is my intention to gradually extend this collection of formulas.
%
\paragraph{Conventions}
The (x.y) and (x.y.z) type subsection numbers, the
(x.y.z) type formula numbers, and the [x] type citation numbers
refer to the book by Koekoek et al.
The (x) type formula numbers refer to this manuscript and the [Kx] type citation numbers refer to citations which are not in the book.
Some standard references like \cite{DLMF} are given by special acronyms.

$N$ is always a positive integer. Always assume $n$ to be a nonnegative
integer or, if $N$ is present, to be in $\{0,1,\ldots,N\}$.
Throughout assume $0<q<1$.

For each family the coefficient of the term of highest degree of the
orthogonal polynomial of degree $n$ can be found in the book as the
coefficient of $p_n(x)$ in the formula after the main formula under
the heading ``Normalized Recurrence Relation". If that main formula is numbered
as (x.y.z) then I will refer to the second formula as (x.y.zb).
%
\subsection*{Generalities}
\paragraph{Critera for uniqueness of orthogonality measure}
According to Shohat \& Tamarkin \cite[p.50]{K6}
orthonormal polynomials $p_n$ have a unique orthogonality measure (up to positive
constant factor) if
for some $z\in\CC$ we have
\begin{equation}
\sum_{n=0}^\iy |p_n(z)|^2 = \iy.
\label{90}
\end{equation}

Also (see Shohat \& Tamarkin \cite[p.59]{K6}),
monic orthogonal polynomials $p_n$ with three-term recurrence relation
$x p_n(x) = p_{n+1}(x)+B_n p_n(x)+C_n p_{n-1}(x)$
($C_n$ necessarily positive)
have a unique orthogonality measure if
\begin{equation}
\sum_{n=1}^\iy (C_n)^{-1/2}=\iy.
\label{93}
\end{equation}

Furthermore, if orthogonal polynomials have an orthogonality measure with
bounded support, then this is unique (see Chihara \cite{146}).
%
\paragraph{Even orthogonality measure}
If $\{p_n\}$ is a system of orthogonal polynomials with respect to an even
orthogonality measure which satisfies the three-term recurrence relation
\begin{equation*}
x p_n(x)=A_n p_{n+1}(x)+C_n p_{n-1}(x)
\end{equation*}
then
\begin{equation}
\frac{p_{2n}(0)}{p_{2n-2}(0)}=-\,\frac{C_{2n-1}}{A_{2n-1}}\,.
\label{1}
\end{equation}
%
\paragraph{Appell's bivariate hypergeometric function $F_4$}
This is defined by
\begin{equation}
F_4(a,b;c,c';x,y):=\sum_{m,n=0}^\iy\frac{(a)_{m+n}b_{m+n}}{(c)_m(c')_n\,m!\,n!}\,
x^my^n\qquad(|x|^\half+|y|^\half<1),
\label{62}
\end{equation}
see \cite[5.7(9), 5.7(44)]{HTF1} or \cite[(16.3.4)]{DLMF}.
There is the reduction formula
\begin{equation*}
F_4\left(a,b;b,b;\frac{-x}{(1-x)(1-y)},\frac{-y}{(1-x)(1-y)}\right)=
(1-x)^a(1-y)^a\,\hyp21{a,1+a-b}b{xy},
\end{equation*}
see \cite[5.10(7)]{HTF1}. When combined with the quadratic transformation
\cite[2.11(34)]{HTF1} (here $a-b-1$ should be replaced by $a-b+1$),
see also \cite[(15.8.15)]{DLMF}, this yields
\begin{multline*}
F_4\left(a,b;b,b;\frac{-x}{(1-x)(1-y)},\frac{-y}{(1-x)(1-y)}\right)\\
=\left(\frac{(1-x)(1-y)}{1+xy}\right)^a\,
\hyp21{\thalf a,\thalf(a+1)}b{\frac{4xy}{(1+xy)^2}}.
\end{multline*}
This can be rewritten as
\begin{equation}
F_4(a,b;b,b;x,y)=(1-x-y)^{-a}\,\hyp21{\thalf a,\thalf(a+1)}b
{\frac{4xy}{(1-x-y)^2}}.
\label{63}
\end{equation}
Note that, if $x,y\ge0$ and $x^\half+y^\half<1$, then
$1-x-y>0$ and $0\le\frac{4xy}{(1-x-y)^2}<1$.
%
\subsection*{9.1 Wilson}
%
\paragraph{Symmetry}
The Wilson polynomial $W_n(y;a,b,c,d)$ is symmetric
in $a,b,c,d$.
\\
This follows from the orthogonality relation (9.1.2)
together with the value of its coefficient of $y^n$ given in (9.1.5b).
Alternatively, combine (9.1.1) with \cite[Theorem 3.1.1]{AAR}.
%
\paragraph{Special value}
\begin{equation}
W_n(-a^2;a,b,c,d)=(a+b)_n(a+c)_n(a+d)_n\,,
\label{91}
\end{equation}
and similarly for arguments $-b^2$, $-c^2$ and
$-d^2$ by symmetry of $W_n$ in $a,b,c,d$.
%
\paragraph{Uniqueness of orthogonality measure}
Under the assumptions on $a,b,c,d$ for (9.1.2) or (9.1.3) the orthogonality
measure is unique up to constant factor.

For the proof assume without
loss of generality (by the symmetry in $a,b,c,d$) that $\Re a\ge0$.
Write the \RHS\ of (9.1.2) or (9.1.3) as $h_n\de_{nm}$.
Observe from (9.1.2) and \eqref{91} that
\[
\frac{|W_n(-a^2;a,b,c,d)|^2}{h_n} = O(n^{4\Re a-1})\quad\hbox{as $n\to\iy$.}
\]
Therefore \eqref{90} holds, from which the uniqueness of the orthogonality
measure follows.

By a similar, but necessarily more complicated argument Ismail et al.\
\cite[Section 3]{281} proved the uniqueness of orthogonality measure for
associated Wilson polynomials.
%
\subsection*{9.3 Continuous dual Hahn}
%
\paragraph{Symmetry}
The continuou dual Hahn polynomial $S_n(y;a,b,c)$ is symmetric
in $a,b,c$.\\
This follows from the orthogonality relation (9.3.2)
together with the value of its coefficient of $y^n$ given in (9.3.5b).
Alternatively, combine (9.3.1) with [??].
%
\paragraph{Special value}
\begin{equation}
S_n(-a^2;a,b,c,d)=(a+b)_n(a+c)_n\,,
\label{92}
\end{equation}
and similarly for arguments $-b^2$ and $-c^2$ by symmetry of $S_n$ in $a,b,c$.
%
\paragraph{Uniqueness of orthogonality measure}
Under the assumptions on $a,b,c$ for (9.3.2) or (9.3.3) the orthogonality
measure is unique up to constant factor.

For the proof assume without
loss of generality (by the symmetry in $a,b,c,d$) that $\Re a\geq0$.
Write the \RHS\ of (9.3.2) or (9.3.3) as $h_n\de_{nm}$.
Observe from (9.3.2) and \eqref{92} that
\[
\frac{|S_n(-a^2;a,b,c)|^2}{h_n} = O(n^{2\Re a-1})\quad
\hbox{as $n\to\iy$.}
\]
Therefore \eqref{90} holds, from which the uniqueness of the orthogonality
measure follows.
%
\subsection*{9.4 Continuous Hahn}
%
\paragraph{Uniqueness of orthogonality measure}
The coefficient of $p_{n-1}(x)$ in (9.4.4) behaves as $O(n^2)$ as $n\to\iy$.
Hence \eqref{93} holds, by which the orthogonality measure is unique.
%
\subsection*{9.5 Hahn}
%
\paragraph{Special values}
From (9.5.3) and \eqref{1} it follows that
\begin{equation}
Q_{2n}(N;\al,\al,2N)=\frac{(\thalf)_n(N+\al+1)_n}{(-N+\thalf)_n(\al+1)_n}\,.
\label{30}
\end{equation}
From (9.5.1) and \cite[(15.4.24)]{DLMF} it follows that
\begin{equation}
Q_N(x;\al,\be,N)=\frac{(-N-\be)_x}{(\al+1)_x}\qquad(x=0,1,\ldots,N).
\label{44}
\end{equation}
%
\paragraph{Duality}
The Remark on p.208 gives the duality between Hahn and dual Hahn polynomials:
%
\begin{equation}
Q_n(x;\al,\be,N)=R_x(n(n+\al+\be+1);\al,\be,N)\quad(n,x\in\{0,1,\ldots N\}).
\label{45}
\end{equation}
%
\subsection*{9.6 Dual Hahn}
%
\paragraph{Re: (9.6.11).}
The generating function (9.6.11) can be written in a more conceptual way as
\begin{equation}
(1-t)^x\,\hyp21{x-N,x+\ga+1}{-\de-N}t=\frac{N!}{(\de+1)_N}\,
\sum_{n=0}^N \om_n\,R_n(\la(x);\ga,\de,N)\,t^n,
\label{2}
\end{equation}
where
\begin{equation}
\om_n:=\binom{\ga+n}n \binom{\de+N-n}{N-n},
\label{3}
\end{equation}
i.e., the denominator on the \RHS\ of (9.6.2).
By the duality between Hahn polynomials and dual Hahn polynomials (see \eqref{45}) the above generating function can be rewritten in
terms of Hahn polynomials:
\begin{equation}
(1-t)^n\,\hyp21{n-N,n+\al+1}{-\be-N}t=\frac{N!}{(\be+1)_N}\,
\sum_{x=0}^N w_x\,Q_n(x;\al,\be,N)\,t^x,
\label{4}
\end{equation}
where
\begin{equation}
w_x:=\binom{\al+x}x \binom{\be+N-x}{N-x},
\label{5}
\end{equation}
i.e., the weight occurring in the orthogonality relation (9.5.2)
for Hahn polynomials.
%
\paragraph{Special value}
By \eqref{44} and \eqref{45} we have
\begin{equation}
R_n(N(N+\ga+\de+1);\ga,\de,N)=\frac{(-N-\de)_n}{(\ga+1)_n}\,.
\label{47}
\end{equation}
%
\subsection*{9.7 Meixner-Pollaczek}
%
\paragraph{Uniqueness of orthogonality measure}
The coefficient of $p_{n-1}(x)$ in (9.7.4) behaves as $O(n^2)$ as $n\to\iy$.
Hence \eqref{93} holds, by which the orthogonality measure is unique.
%
\subsection*{9.8 Jacobi}
%
\paragraph{Orthogonality relation}
Write the \RHS\ of (9.8.2) as $h_n\,\de_{m,n}$. Then
\begin{equation}
\frac{h_n}{h_0}=
\frac{n+\al+\be+1}{2n+\al+\be+1}\,
\frac{(\al+1)_n(\be+1)_n}{(\al+\be+2)_n\,n!}\,.
\label{60}
\end{equation}
%
\paragraph{Symmetry}
\begin{equation}
P_n^{(\al,\be)}(-x)=(-1)^n\,P_n^{(\be,\al)}(x).
\label{48}
\end{equation}
Use (9.8.2) and (9.8.5b) or see \cite[Table 18.6.1]{DLMF}.
%
\paragraph{Special values}
\begin{equation}
P_n^{(\al,\be)}(1)=\frac{(\al+1)_n}{n!}\,,\quad
P_n^{(\al,\be)}(-1)=\frac{(-1)^n(\be+1)_n}{n!}\,.
\label{50}
\end{equation}
Use (9.8.1) and \eqref{48} or see \cite[Table 18.6.1]{DLMF}.
%
\paragraph{Bilateral generating functions}
For $0\le r<1$ and $x,y\in[-1,1]$ we have in terms of $F_4$ (see~\eqref{62}):
\begin{align}
&\sum_{n=0}^\iy\frac{(\al+\be+1)_n\,n!}{(\al+1)_n(\be+1)_n}\,r^n\,
P_n^{(\al,\be)}(x)\,P_n^{(\al,\be)}(y)
=\frac1{(1+r)^{\al+\be+1}}
\nonumber\\
&\qquad\quad\times F_4\Big(\thalf(\al+\be+1),\thalf(\al+\be+2);\al+1,\be+1;
\frac{r(1-x)(1-y)}{(1+r)^2},\frac{r(1+x)(1+y)}{(1+r)^2}\Big),
\label{58}\sLP
&\sum_{n=0}^\iy\frac{2n+\al+\be+1}{n+\al+\be+1}
\frac{(\al+\be+2)_n\,n!}{(\al+1)_n(\be+1)_n}\,r^n\,
P_n^{(\al,\be)}(x)\,P_n^{(\al,\be)}(y)
=\frac{1-r}{(1+r)^{\al+\be+2}}\nonumber\\
&\qquad\quad\times F_4\Big(\thalf(\al+\be+2),\thalf(\al+\be+3);\al+1,\be+1;
\frac{r(1-x)(1-y)}{(1+r)^2},\frac{r(1+x)(1+y)}{(1+r)^2}\Big).
\label{59}
\end{align}
Formulas \eqref{58} and \eqref{59} were first
given by Bailey \cite[(2.1), (2.3)]{91}.
See Stanton \cite{485} for a shorter proof. (However, in the second line of
\cite[(1)]{485} $z$ and $Z$ should be interchanged.)$\;$
As observed in Bailey \cite[p.10]{91}, \eqref{59} follows from \eqref{58}
by applying the operator $r^{-\half(\al+\be-1)}\,\frac d{dr}\circ r^{\half(\al+\be+1)}$
to both sides of \eqref{58}.
In view of \eqref{60}, formula \eqref{59} is the Poisson kernel for Jacobi
polynomials. The \RHS\ of \eqref{59} makes clear that this kernel is positive.
See also the discussion in Askey \cite[following (2.32)]{46}.
%
\paragraph{Quadratic transformations}
\begin{align}
\frac{C_{2n}^{(\al+\half)}(x)}{C_{2n}^{(\al+\half)}(1)}
=\frac{P_{2n}^{(\al,\al)}(x)}{P_{2n}^{(\al,\al)}(1)}
&=\frac{P_n^{(\al,-\half)}(2x^2-1)}{P_n^{(\al,-\half)}(1)}\,,
\label{51}\\
\frac{C_{2n+1}^{(\al+\half)}(x)}{C_{2n+1}^{(\al+\half)}(1)}
=\frac{P_{2n+1}^{(\al,\al)}(x)}{P_{2n+1}^{(\al,\al)}(1)}
&=\frac{x\,P_n^{(\al,\half)}(2x^2-1)}{P_n^{(\al,\half)}(1)}\,.
\label{52}
\end{align}
See p.221, Remarks, last two formulas together with \eqref{50} and \eqref{49}.
Or see \cite[(18.7.13), (18.7.14)]{DLMF}.
%
\paragraph{Differentiation formulas}
Each differentiation formula is given in two equivalent forms.
\begin{equation}
\begin{split}
\frac d{dx}\left((1-x)^\al P_n^{(\al,\be)}(x)\right)&=
-(n+\al)\,(1-x)^{\al-1} P_n^{(\al-1,\be+1)}(x),\\
\left((1-x)\frac d{dx}-\al\right)P_n^{(\al,\be)}(x)&=
-(n+\al)\,P_n^{(\al-1,\be+1)}(x).
\end{split}
\label{68}
\end{equation}
%
\begin{equation}
\begin{split}
\frac d{dx}\left((1+x)^\be P_n^{(\al,\be)}(x)\right)&=
(n+\be)\,(1+x)^{\be-1} P_n^{(\al+1,\be-1)}(x),\\
\left((1+x)\frac d{dx}+\be\right)P_n^{(\al,\be)}(x)&=
(n+\be)\,P_n^{(\al+1,\be-1)}(x).
\end{split}
\label{69}
\end{equation}
Formulas \eqref{68} and \eqref{69} follow from \cite[(15.5.4), (15.5.6)]{DLMF}
together with (9.8.1). They also follow from each other by \eqref{48}.
%
\paragraph{Generalized Gegenbauer polynomials}
See \cite[p.156]{146}.
These are defined by
\begin{equation}
S_{2m}^{(\al,\be)}(x):=\const P_m^{(\al,\be)}(2x^2-1),\qquad
S_{2m+1}^{(\al,\be)}(x):=\const x\,P_m^{(\al,\be+1)}(2x^2-1).
\label{70}
\end{equation}
Then for $\al,\be>-1$ we have the orthogonality relation
\begin{equation}
\int_{-1}^1 S_m^{(\al,\be)}(x)\,S_n^{(\al,\be)}(x)\,|x|^{2\be+1}(1-x^2)^\al\,dx
=0\qquad(m\ne n).
\label{71}
\end{equation}
If we define the {\em Dunkl operator} $T_\mu$ by
\begin{equation}
(T_\mu f)(x):=f'(x)+\mu\,\frac{f(x)-f(-x)}x
\label{72}
\end{equation}
and if we choose the constants in \eqref{70} as
\begin{equation}
S_{2m}^{(\al,\be)}(x)=\frac{(\al+\be+1)_m}{(\be+1)_m}\, P_m^{(\al,\be)}(2x^2-1),\quad
S_{2m+1}^{(\al,\be)}(x)=\frac{(\al+\be+1)_{m+1}}{(\be+1)_{m+1}}\,
x\,P_m^{(\al,\be+1)}(2x^2-1)
\label{73}
\end{equation}
then (see \cite[(1.6)]{K5})
\begin{equation}
T_{\be+\half}S_n^{(\al,\be)}=2(\al+\be+1)\,S_{n-1}^{(\al+1,\be)}.
\label{74}
\end{equation}
Formula \eqref{74} with \eqref{73} substituted gives rise to two
differentiation formulas involving Jacobi polynomials which are equivalent to
(9.8.7) and \eqref{69}.

Composition of \eqref{74} with itself gives
\[
T_{\be+\half}^2S_n^{(\al,\be)}=4(\al+\be+1)(\al+\be+2)\,S_{n-2}^{(\al+2,\be)},
\]
which is equivalent to the composition of (9.8.7) and \eqref{69}:
\begin{equation}
\left(\frac{d^2}{dx^2}+\frac{2\be+1}x\,\frac d{dx}\right)P_n^{(\al,\be)}(2x^2-1)
=4(n+\al+\be+1)(n+\be)\,P_{n-1}^{(\al+2,\be)}(2x^2-1).
\label{75}
\end{equation}
Formula \eqref{75} was also given in \cite[(2.4)]{322}.
%
\subsection*{9.8.1 Gegenbauer / Ultraspherical}
%
\paragraph{Notation}
Here the Gegenbauer polynomial is denoted by $C_n^\la$ instead of $C_n^{(\la)}$.
%
\paragraph{Orthogonality relation}
Write the \RHS\ of (9.8.20) as $h_n\,\de_{m,n}$. Then
\begin{equation}
\frac{h_n}{h_0}=
\frac\la{\la+n}\,\frac{(2\la)_n}{n!}\,.
\label{61}
\end{equation}
%
\paragraph{Hypergeometric representation}
Beside (9.8.19) we have also
\begin{equation}
C_n^\lambda(x)=\sum_{\ell=0}^{\lfloor n/2\rfloor}\frac{(-1)^{\ell}(\lambda)_{n-\ell}}
{\ell!\;(n-2\ell)!}\,(2x)^{n-2\ell}
=(2x)^{n}\,\frac{(\lambda)_{n}}{n!}\,
\hyp21{-\thalf n,-\thalf n+\thalf}{1-\la-n}{\frac1{x^2}}.
\label{57}
\end{equation}
See \cite[(18.5.10)]{DLMF}.
%
\paragraph{Special value}
\begin{equation}
C_n^{\la}(1)=\frac{(2\la)_n}{n!}\,.
\label{49}
\end{equation}
Use (9.8.19) or see \cite[Table 18.6.1]{DLMF}.
%
\paragraph{Expression in terms of Jacobi}
%
\begin{equation}
\frac{C_n^\la(x)}{C_n^\la(1)}=
\frac{P_n^{(\la-\half,\la-\half)}(x)}{P_n^{(\la-\half,\la-\half)}(1)}\,,\qquad
C_n^\la(x)=\frac{(2\la)_n}{(\la+\thalf)_n}\,P_n^{(\la-\half,\la-\half)}(x).
\label{65}
\end{equation}
%
\paragraph{Re: (9.8.21)}
By iteration of recurrence relation (9.8.21):
\begin{multline}
x^2 C_n^\la(x)=
\frac{(n+1)(n+2)}{4(n+\la)(n+\la+1)}\,C_{n+2}^\la(x)+
\frac{n^2+2n\la+\la-1}{2(n+\la-1)(n+\la+1)}\,C_n^\la(x)\\
+\frac{(n+2\la-1)(n+2\la-2)}{4(n+\la)(n+\la-1)}\,C_{n-2}^\la(x).
\label{6}
\end{multline}
%
\paragraph{Bilateral generating functions}
\begin{multline}
\sum_{n=0}^\iy\frac{n!}{(2\la)_n}\,r^n\,C_n^\la(x)\,C_n^\la(y)
=\frac1{(1-2rxy+r^2)^\la}\,\hyp21{\thalf\la,\thalf(\la+1)}{\la+\thalf}
{\frac{4r^2(1-x^2)(1-y^2)}{(1-2rxy+r^2)^2}}\\
(r\in(-1,1),\;x,y\in[-1,1]).
\label{66}
\end{multline}
For the proof put $\be:=\al$ in \eqref{58}, then use \eqref{63} and \eqref{65}.
The Poisson kernel for Gegenbauer polynomials can be derived in a similar way
from \eqref{59}, or alternatively by applying the operator
$r^{-\la+1}\frac d{dr}\circ r^\la$ to both sides of \eqref{66}:
\begin{multline}
\sum_{n=0}^\iy\frac{\la+n}\la\,\frac{n!}{(2\la)_n}\,r^n\,C_n^\la(x)\,C_n^\la(y)
=\frac{1-r^2}{(1-2rxy+r^2)^{\la+1}}\\
\times\hyp21{\thalf(\la+1),\thalf(\la+2)}{\la+\thalf}
{\frac{4r^2(1-x^2)(1-y^2)}{(1-2rxy+r^2)^2}}\qquad
(r\in(-1,1),\;x,y\in[-1,1]).
\label{67}
\end{multline}
Formula \eqref{67} was obtained by Gasper \& Rahman \cite[(4.4)]{234}
as a limit case of their formula for the Poisson kernel for continuous
$q$-ultraspherical polynomials.
%
\paragraph{A trigonometric expansion}
By \cite[(14.13.1), (14.3.21), (5.5.5)]{DLMF}:
\begin{align}
C_n^\la(\cos\tha)=\frac{\Ga(2\la+1)}{2^{2\la}\Ga(\la+1)^2}\,
\frac{(2\la)_n}{(\la+1)_n}\,(\sin\tha)^{1-2\la}\,
\sum_{k=0}^\iy\frac{(1-\la)_k(n+1)_k}{(n+\la+1)_k k!}\,
\sin\big((2k+n+1)\tha\big)
\label{7}\\
(\la>0,\;0<\tha<\pi).
\nonumber
\end{align}
For $\la\in\Zpos$ the above series terminates after the term with
$k=\la-1$.
%
\paragraph{Fourier transform}
\begin{equation}
\frac{\Ga(\la+1)}{\Ga(\la+\thalf)\,\Ga(\thalf)}\,
\int_{-1}^1 \frac{C_n^\la(y)}{C_n^\la(1)}\,(1-y^2)^{\la-\half}\,
e^{ixy}\,dy
=i^n\,2^\la\,\Ga(\la+1)\,x^{-\la}\,J_{\la+n}(x).
\label{8}
\end{equation}
See \cite[(18.17.17) and (18.17.18)]{DLMF}.
%
\paragraph{Laplace transforms}
\begin{equation}
\frac2{n!\,\Ga(\la)}\,
\int_0^\iy H_n(tx)\,t^{n+2\la-1}\,e^{-t^2}\,dt=C_n^\la(x).
\label{56}
\end{equation}
See Nielsen \cite[p.48, (4) with p.47, (1) and p.28, (10)]{K4} (1918)
or Feldheim \cite[(28)]{K3} (1942).
\begin{equation}
\frac2{\Ga(\la+\thalf)}\,\int_0^1 \frac{C_n^\la(t)}{C_n^\la(1)}\,
(1-t^2)^{\la-\half}\,t^{-1}\,(x/t)^{n+2\la+1}\,e^{-x^2/t^2}\,dt
=2^{-n}\,H_n(x)\,e^{-x^2}\quad(\la>-\thalf).
\label{46}
\end{equation}
Use Askey \& Fitch \cite[(3.29)]{K2} for $\al=\pm\thalf$ together with
\eqref{48}, \eqref{51}, \eqref{52}, \eqref{54} and \eqref{55}.
%
\subsection*{9.10 Meixner}
\paragraph{History}
In 1934 Meixner \cite{406} (see
(1.1) and case IV on pp.~10, 11 and 12) gave the orthogonality
measure for the polynomials $P_n$ given by the generating function
\[
e^{x u(t)}\,f(t)=\sum_{n=0}^\iy P_n(x)\,\frac{t^n}{n!}\,,
\]
where
\[
e^{u(t)}=\left(\frac{1-\be t}{1-\al t}\right)^{\frac1{\al-\be}},\quad
f(t)=\frac{(1-\be t)^{\frac{k_2}{\be(\al-\be)}}}{(1-\al t)^{\frac{k_2}{\al(\al-\be)}}}\quad
(k_2<0;\;\al>\be>0\;\;{\rm or}\;\;\al<\be<0).
\]
Then $P_n$ can be expressed as a Meixner polynomial:
\[
P_n(x)=(-k_2(\al\be)^{-1})_n\,\be^n\,
M_n\left(-\,\frac{x+k_2\al^{-1}}{\al-\be},-k_2(\al\be)^{-1},\be\al^{-1}\right).
\]

In 1938 Gottlieb \cite[\S2]{K1} introduces polynonials $l_n$ ``of Laguerre type''
which turn out to be special Meixner polynomials:
$l_n(x)=e^{-n\la} M_n(x;1,e^{-\la})$.
%
\paragraph{Uniqueness of orthogonality measure}
The coefficient of $p_{n-1}(x)$ in (9.10.4) behaves as $O(n^2)$ as $n\to\iy$.
Hence \eqref{93} holds, by which the orthogonality measure is unique.
%
\subsection*{9.11 Krawtchouk}
%
\paragraph{Special values}
By (9.11.1) and the binomial formula:
\begin{equation}
K_n(0;p,N)=1,\qquad
K_n(N;p,N)=(-1)^n p^{-n}(1-p)^n.
\label{9}
\end{equation}
%
\paragraph{Symmetry}
By the orthogonality relations:
\begin{equation}
\frac{K_n(N-x;p,N)}{K_n(N;p,N)}=K_n(x;1-p,N),
\label{10}
\end{equation}
in particular:
\begin{equation}
K_n(N-x;\thalf,N)=(-1)^n K_n(x;\thalf,N).
\label{11}
\end{equation}
Hence
\begin{equation}
K_{2m+1}(N;\thalf,2N)=0.
\label{12}
\end{equation}
From (9.11.11):
\begin{equation}
K_{2m}(N;\thalf,2N)=\frac{(\thalf)_m}{(-N+\thalf)_m}\,.
\label{13}
\end{equation}
%
\paragraph{Quadratic transformations}
\begin{align}
K_{2m}(x+N;\thalf,2N)&=\frac{(\thalf)_m}{(-N+\thalf)_m}\,
R_m(x^2;-\thalf,-\thalf,N),
\label{31}\\
K_{2m+1}(x+N;\thalf,2N)&=-\,\frac{(\tfrac32)_m}{N\,(-N+\thalf)_m}\,
x\,R_m(x^2-1;\thalf,\thalf,N-1),
\label{33}\\
K_{2m}(x+N+1;\thalf,2N+1)&=\frac{(\tfrac12)_m}{(-N-\thalf)_m}\,
R_m(x(x+1);-\thalf,\thalf,N),
\label{32}\\
K_{2m+1}(x+N+1;\thalf,2N+1)&=\frac{(\tfrac32)_m}{(-N-\thalf)_{m+1}}\,
(x+\thalf)\,R_m(x(x+1);\thalf,-\thalf,N),
\label{34}
\end{align}
where $R_m$ is a dual Hahn polynomial (9.6.1). For the proofs use
(9.6.2), (9.11.2), (9.6.4) and (9.11.4).
%
\subsection*{9.12 Laguerre}
\paragraph{Notation}
Here the Laguerre polynomial is denoted by $L_n^\al$ instead of
$L_n^{(\al)}$.
%
\paragraph{Uniqueness of orthogonality measure}
The coefficient of $p_{n-1}(x)$ in (9.12.4) behaves as $O(n^2)$ as $n\to\iy$.
Hence \eqref{93} holds, by which the orthogonality measure is unique.
%
\paragraph{Special value}
\begin{equation}
L_n^{\al}(0)=\frac{(\al+1)_n}{n!}\,.
\label{53}
\end{equation}
Use (9.12.1) or see \cite[(18.6.1)]{DLMF}.
%
\paragraph{Quadratic transformations}
\begin{align}
H_{2n}(x)&=(-1)^n\,2^{2n}\,n!\,L_n^{-1/2}(x^2),
\label{54}\\
H_{2n+1}(x)&=(-1)^n\,2^{2n+1}\,n!\,x\,L_n^{1/2}(x^2).
\label{55}
\end{align}
See p.244, Remarks, last two formulas.
Or see \cite[(18.7.19), (18.7.20)]{DLMF}.
%
\paragraph{Fourier transform}
\begin{equation}
\frac1{\Ga(\al+1)}\,\int_0^\iy \frac{L_n^\al(y)}{L_n^\al(0)}\,
e^{-y}\,y^\al\,e^{ixy}\,dy=
i^n\,\frac{y^n}{(iy+1)^{n+\al+1}}\,,
\label{14}
\end{equation}
see \cite[(18.17.34)]{DLMF}.
%
\paragraph{Differentiation formulas}
Each differentiation formula is given in two equivalent forms.
\begin{equation}
\frac d{dx}\left(x^\al L_n^\al(x)\right)=
(n+\al)\,x^{\al-1} L_n^{\al-1}(x),\qquad
\left(x\frac d{dx}+\al\right)L_n^\al(x)=
(n+\al)\,L_n^{\al-1}(x).
\label{76}
\end{equation}
%
\begin{equation}
\frac d{dx}\left(e^{-x} L_n^\al(x)\right)=
-e^{-x} L_n^{\al+1}(x),\qquad
\left(\frac d{dx}-1\right)L_n^\al(x)=
-L_n^{\al+1}(x).
\label{77}
\end{equation}
%
Formulas \eqref{76} and \eqref{77} follow from \cite[(13.3.18), (13.3.20)]{DLMF}
together with (9.12.1). 
%
\paragraph{Generalized Hermite polynomials}
See \cite[p.156]{146}.
These are defined by
\begin{equation}
H_{2m}^\mu(x):=\const L_m^{\mu-\half}(x^2),\qquad
H_{2m+1}^\mu(x):=\const x\,L_m^{\mu+\half}(x^2).
\label{78}
\end{equation}
Then for $\mu>-\thalf$ we have orthogonality relation
\begin{equation}
\int_{-\iy}^{\iy} H_m^\mu(x)\,H_n^\mu(x)\,|x|^{2\mu}e^{-x^2}\,dx
=0\qquad(m\ne n).
\label{79}
\end{equation}
Let the Dunkl operator $T_\mu$ be defined by \eqref{72}.
If we choose the constants in \eqref{78} as
\begin{equation}
H_{2m}^\mu(x)=\frac{(-1)^m(2m)!}{(\mu+\thalf)_m}\,L_m^{\mu-\half}(x^2),\qquad
H_{2m+1}^\mu(x)=\frac{(-1)^m(2m+1)!}{(\mu+\thalf)_{m+1}}\,
 x\,L_m^{\mu+\half}(x^2)
 \label{80}
\end{equation}
then (see \cite[(1.6)]{K5})
\begin{equation}
T_\mu H_n^\mu=2n\,H_{n-1}^\mu.
\label{81}
\end{equation}
Formula \eqref{81} with \eqref{80} substituted gives rise to two
differentiation formulas involving Laguerre polynomials which are equivalent to
(9.12.6) and \eqref{76}.

Composition of \eqref{81} with itself gives
\[
T_\mu^2 H_n^\mu=4n(n-1)\,H_{n-2}^\mu,
\]
which is equivalent to the composition of (9.12.6) and \eqref{76}:
\begin{equation}
\left(\frac{d^2}{dx^2}+\frac{2\al+1}x\,\frac d{dx}\right)L_n^\al(x^2)
=-4(n+\al)\,L_{n-1}^\al(x^2).
\label{82}
\end{equation}
%
\subsection*{9.14 Charlier}
%
\paragraph{Uniqueness of orthogonality measure}
The coefficient of $p_{n-1}(x)$ in (9.14.4) behaves as $O(n)$ as $n\to\iy$.
Hence \eqref{93} holds, by which the orthogonality measure is unique.
%
\subsection*{9.15 Hermite}
%
\paragraph{Uniqueness of orthogonality measure}
The coefficient of $p_{n-1}(x)$ in (9.15.4) behaves as $O(n)$ as $n\to\iy$.
Hence \eqref{93} holds, by which the orthogonality measure is unique.
%
\paragraph{Fourier transforms}
\begin{equation}
\frac1{\sqrt{2\pi}}\,\int_{-\iy}^\iy H_n(y)\,e^{-\half y^2}\,e^{ixy}\,dy=
i^n\,H_n(x)\,e^{-\half x^2},
\label{15}
\end{equation}
see \cite[(6.1.15) and Exercise 6.11]{AAR}.
\begin{equation}
\frac1{\sqrt\pi}\,\int_{-\iy}^\iy H_n(y)\,e^{-y^2}\,e^{ixy}\,dy=
i^n\,x^n\,e^{-\frac14 x^2},
\label{16}
\end{equation}
see \cite[(18.17.35)]{DLMF}.
\begin{equation}
\frac{i^n}{2\sqrt\pi}\,\int_{-\iy}^\iy y^n\,e^{-\frac14 y^2}\,e^{-ixy}\,dy=
H_n(x)\,e^{-x^2},
\label{17}
\end{equation}
see \cite[(6.1.4)]{AAR}.
%
\subsection*{14.1 Askey-Wilson}
%
\paragraph{Symmetry}
The Askey-Wilson polynomials $p_n(x;a,b,c,d\mid q)$ are symmetric
in $a,b,c,d$.
\sLP
This follows from the orthogonality relation (14.1.2)
together with the value of its coefficient of $x^n$ given in (14.1.5b).
Alternatively, combine (14.1.1) with \cite[(III.15)]{GR}.
%
\paragraph{Special value}
\begin{equation}
p_n\big(\thalf(a+a^{-1});a,b,c,d\mid q\big)=a^{-n}\,(ab,ac,ad;q)_n\,,
\label{40}
\end{equation}
and similarly for arguments $\thalf(b+b^{-1})$, $\thalf(c+c^{-1})$ and
$\thalf(d+d^{-1})$ by symmetry of $p_n$ in $a,b,c,d$.
%
\paragraph{Trivial symmetry}
\begin{equation}
p_n(-x;a,b,c,d\mid q)=(-1)^n p_n(x;-a,-b,-c,-d\mid q).
\label{41}
\end{equation}
Both \eqref{40} and \eqref{41} are obtained from (14.1.1).
%
\paragraph{Re: (14.1.5)}
Let
\begin{equation}
p_n(x):=\frac{p_n(x;a,b,c,d\mid q)}{2^n(abcdq^{n-1};q)_n}=x^n+\wt k_n x^{n-1}
+\cdots\;.
\label{18}
\end{equation}
Then
\begin{equation}
\wt k_n=-\frac{(1-q^n)(a+b+c+d-(abc+abd+acd+bcd)q^{n-1})}
{2(1-q)(1-abcdq^{2n-2})}\,.
\label{19}
\end{equation}
This follows because $\tilde k_n-\tilde k_{n+1}$ equals the coefficient
$\thalf\bigl(a+a^{-1}-(A_n+C_n)\bigr)$ of $p_n(x)$ in (14.1.5).
%
\subsection*{14.2 $q$-Racah}
\paragraph{Symmetry}
\begin{equation}
R_n(x;\al,\be,q^{-N-1},\de\mid q)
=\frac{(\be q,\al\de^{-1}q;q)_n}{(\al q,\be\de q;q)_n}\,\de^n\,
R_n(\de^{-1}x;\be,\al,q^{-N-1},\de^{-1}\mid q).
\label{84}
\end{equation}
This follows from (14.2.1) combined with \cite[(III.15)]{GR}.
\sLP
In particular,
\begin{equation}
R_n(x;\al,\be,q^{-N-1},-1\mid q)
=\frac{(\be q,-\al q;q)_n}{(\al q,-\be q;q)_n}\,(-1)^n\,
R_n(-x;\be,\al,q^{-N-1},-1\mid q),
\label{85}
\end{equation}
and
\begin{equation}
R_n(x;\al,\al,q^{-N-1},-1\mid q)
=(-1)^n\,R_n(-x;\al,\al,q^{-N-1},-1\mid q),
\label{86}
\end{equation}

\paragraph{Trivial symmetry}
Clearly from (14.2.1):
\begin{equation}
R_n(y;\al,\be,\ga,\de\mid q)=R_n(y;\be\de,\al\de^{-1},\ga,\de\mid q).
\label{83}
\end{equation}
%
\subsection*{14.7 Dual $q$-Hahn}
\paragraph{Orthogonality relation}
More generally we have (14.7.2) with positive weights in any of the following
cases:
(i) $0<\ga q<1$, $0<\de q<1$;\quad
(ii) $0<\ga q<1$, $\de<0$;\quad
(iii) $\ga<0$, $\de>q^{-N}$;\quad
(iv) $\ga>q^{-N}$, $\de>q^{-N}$;\quad
(v) $0<q\ga<1$, $\de=0$.
This also follows by inspection of the positivity of the coefficient of
$p_{n-1}(x)$ in (14.7.4).
Case (v) yields Affine $q$-Krawtchouk in view of (14.7.13).
%
\paragraph{Symmetry}
\begin{equation}
R_n(x;\ga,\de,N\mid q)
=\frac{(\de^{-1}q^{-N};q)_n}{(\ga q;q)_n}\,\big(\ga\de q^{N+1}\big)^n\,
R_n(\ga^{-1}\de^{-1}q^{-1-N} x;\de^{-1}q^{-N-1},\ga^{-1}q^{-N-1},N\mid q).
\label{89}
\end{equation}
This follows from (14.7.1) combined with \cite[(III.11)]{GR}.
%
\subsection*{14.8 Al-Salam-Chihara}
\subsubsection*{$q^{-1}$-Al-Salam-Chihara}
%
\paragraph{Re: (14.8.1)}
For $x\in\Znonneg$:
%
\begin{align}
Q_n(\thalf(aq^{-x}+a^{-1}q^x);&a,b\mid q^{-1})=
(-1)^n b^n q^{-\half n(n-1)}\left((ab)^{-1};q\right)_n
\nonumber\\
&\qquad\qquad\qquad\qquad\qquad\quad
\times\qhyp31{q^{-n},q^{-x},a^{-2}q^x}{(ab)^{-1}}{q,q^nab^{-1}}
\label{20}\\
&=(-ab^{-1})^x\,q^{-\half x(x+1)}\,\frac{(qba^{-1};q)_x}{(a^{-1}b^{-1};q)_x}\,
\qhyp21{q^{-x},a^{-2}q^x}{qba^{-1}}{q,q^{n+1}}
\label{42}\\
&=(-ab^{-1})^x\,q^{-\half x(x+1)}\,\frac{(qba^{-1};q)_x}{(a^{-1}b^{-1};q)_x}\,
p_x(q^n;ba^{-1},(qab)^{-1};q).
\label{43}
\end{align}
%
Formula \eqref{20} follows from the first identity in (14.8.1).
Next \eqref{42} follows from \cite[(III.8)]{GR}.
Finally \eqref{43} gives the little $q$-Jacobi polynomials (14.12.1).
See also \cite[\S3]{79}.
%
\paragraph{Orthogonality}
%
\begin{multline}
\sum_{x=0}^\iy
\frac{(1-q^{2x}a^{-2}) (a^{-2},(ab)^{-1};q)_x}
{(1-a^{-2}) (q,bqa^{-1};q)_x}\,
(ba^{-1})^xq^{x^2}
(Q_nQ_m)(\thalf(aq^{-x}+a^{-1}q^x);a,b;q)\\
=\frac{(qa^{-2};q)_\iy}{(ba^{-1}q;q)_\iy}\,
(q,(ab)^{-1};q)_n\,(ab)^nq^{-n^2}\,\de_{n,m}
\quad(ab>1,\;qb<a).
\label{21}
\end{multline}
%
This follows from (29) together with (14.12.2) and the completeness of
the orthogonal systerm of the little $q$-Jacobi polynomials,
See also \cite[\S3]{79}. An alternative proof is given in
\cite{64}. There combine (3.82) with (3.81), (3.67), (3.40).
%
\paragraph{Normalized recurrence relation}
%
\begin{equation}
xp_n(x)=p_{n+1}(x)+\thalf(a+b)q^{-n} p_n(x)+
\tfrac14(q^{-n}-1)(abq^{-n+1}-1)p_{n-1}(x),
\label{22}
\end{equation}
%
where
\[
Q_n(x;a,b\mid q^{-1})=2^n p_n(x).
\]
%
\subsection*{14.10.1 Continuous $q$-ultraspherical / Rogers}
\paragraph{Re: (14.10.17)}
\begin{equation}
C_n(\cos\tha;\be\mid q)=
\frac{(\be^2;q)_n}{(q;q)_n}\,\be^{-\half n}\,
\qhyp43{q^{-\half n},\be q^{\half n},\be^\half e^{i\tha},\be^\half e^{-i\tha}}
{-\be,\be^\half q^{\frac14},-\be^\half q^{\frac14}}{q^\half,q^\half},
\label{23}
\end{equation}
see \cite[(7.4.13), (7.4.14)]{GR}.
%
\paragraph{Re: (14.10.21)}
(another $q$-difference equation).
Let $C_n[e^{i\tha};\be\mid q]:=C_n(\cos\tha;\be\mid q)$.
\begin{equation}
\frac{1-\be z^2}{1-z^2}\,C_n[q^\half z;\be\mid q]+
\frac{1-\be z^{-2}}{1-z^{-2}}\,C_n[q^{-\half}z;\be\mid q]=
(q^{-\half n}+q^{\half n} \be)\,C_n[z;\be\mid q],
\label{24}
\end{equation}
see \cite[(6.10)]{351}.
%
\paragraph{Re: (14.10.23)}
This can also be written as
\begin{equation}
C_n[q^\half z;\be\mid q]-C_n[q^{-\half}z;\be\mid q]=
q^{-\half n}(\be-1)(z-z^{-1})C_{n-1}[z;q\be\mid q].
\label{25}
\end{equation}
Two other shift relations follow from the previous two equations:
\begin{align}
(\be+1)C_n[q^\half z;\be\mid q]&=(q^{-\half n}+q^{\half n}\be)C_n[z;\be\mid q]
+q^{-\half n}(\be-1)(z-\be z^{-1})C_{n-1}[z;q\be\mid q],
\label{26}\\
(\be+1)C_n[q^{-\half}z;\be\mid q]&=(q^{-\half n}+q^{\half n}\be)C_n[z;\be\mid q]
+q^{-\half n}(\be-1)(z^{-1}-\be z)C_{n-1}[z;q\be\mid q].
\label{27}
\end{align}
%
\subsection*{14.17 Dual $q$-Krawtchouk}
%
\paragraph{Symmetry}
\begin{equation}
K_n(x;c,N\mid q)=c^n\,K_n(c^{-1}x;c^{-1},N\mid q).
\label{87}
\end{equation}
This follows from (14.17.1) combined with \cite[(III.11)]{GR}.
\sLP
In particular,
\begin{equation}
K_n(x;-1,N\mid q)=(-1)^n\,K_n(-x;-1,N\mid q).
\label{88}
\end{equation}
%
\subsection*{14.20 Little $q$-Laguerre / Wall}
%
\paragraph{Re: (14.20.11)}
The \RHS\ of this generating function converges for $|xt|<1$.
We can rewrite the \LHS\ by use of the transformation
\begin{equation*}
\qhyp21{0,0}c{q,z}=\frac1{(z;q)_\iy}\,\qhyp01-c{q,cz}.
\end{equation*}
Then we obtain:
\begin{equation}
(t;q)_\iy\,\qhyp21{0,0}{aq}{q,xt}
=\sum_{n=0}^\iy\frac{(-1)^n\,q^{\half n(n-1)}}{(q;q)_n}\,
p_n(x;a;q)\,t^n\qquad(|xt|<1).
\label{35}
\end{equation}
%
\subsubsection*{Expansion of $x^n$}
Divide both sides of \eqref{35} by $(t;q)_\iy$. Then coefficients of the
same power of $t$ on both sides must be equal. We obtain:
\begin{equation}
x^n=(a;q)_n\,\sum_{k=0}^n \frac{(q^{-n};q)_k}{(q;q)_k}\,q^{nk}\,p_k(x;a;q).
\label{36}
\end{equation}
%
\subsubsection*{Quadratic transformations}
Little $q$-Laguerre polynomials $p_n(x;a;q)$ with $a=q^{\pm\half}$ are
related to discrete $q$-Hermite I polynomials $h_n(x;q)$:
\begin{align}
p_n(x^2;q^{-1};q^2)&=
\frac{(-1)^n q^{-n(n-1)}}{(q;q^2)_n}\,h_{2n}(x;q),
\label{28}\\
xp_n(x^2;q;q^2)&=
\frac{(-1)^n q^{-n(n-1)}}{(q^3;q^2)_n}\,h_{2n+1}(x;q).
\label{29}
\end{align}
%
\subsection*{14.21 $q$-Laguerre}
%
\subsubsection*{Expansion of $x^n$}
\begin{equation}
x^n=q^{-\half n(n+2\al+1)}\,(q^{\al+1};q)_n\,
\sum_{k=0}^n\frac{(q^{-n};q)_k}{q^{\al+1};q)_k}\,q^k\,L_k^\al(x;q).
\label{37}
\end{equation}
This follows from \eqref{36} by the equality given in the Remark at the end
of \S14.20. Alternatively, it can be derived in the same way as \eqref{36}
from the generating function (14.21.14).
%
\subsubsection*{Quadratic transformations}
$q$-Laguerre polynomials $L_n^\al(x;q)$ with $\al=\pm\half$ are
related to discrete $q$-Hermite II polynomials $\wt h_n(x;q)$:
\begin{align}
L_n^{-1/2}(x^2;q^2)&=
\frac{(-1)^n q^{2n^2-n}}{(q^2;q^2)_n}\,\wt h_{2n}(x;q),
\label{38}\\
xL_n^{1/2}(x^2;q^2)&=
\frac{(-1)^n q^{2n^2+n}}{(q^2;q^2)_n}\,\wt h_{2n+1}(x;q).
\label{39}
\end{align}
These follows from \eqref{28} and \eqref{29}, respectively, by applying
the equalities given in the Remarks at the end of \S14.20 and \S14.28.
%
\renewcommand{\refname}{Standard references}
\begin{thebibliography}{9999999}
%
\bibitem[AAR]{AAR}
G. E. Andrews, R. Askey and R. Roy,
{\em Special functions},
Cambridge University Press, 1999.
%
\bibitem[DLMF]{DLMF}
{\em NIST Handbook of Mathematical Functions},
Cambridge University Press, 2010;\\
{\em DLMF, Digital Library of Mathematical Functions},
\url{http://dlmf.nist.gov}.
%
\bibitem[GR]{GR}
G.~Gasper and M.~Rahman,
{\em Basic hypergeometric series}, 2nd edn.,
Cambridge University Press, 2004.
%
\bibitem[HTF1]{HTF1}
A. Erd\'elyi,
{\em Higher transcendental functions, Vol. 1},
McGraw-Hill, 1953.
%
\end{thebibliography}
%
\renewcommand{\refname}{References from Koekoek, Lesky \& Swarttouw}
\begin{thebibliography}{999}
%
\bibitem[46]{46}
R. Askey,
{\em Orthogonal polynomials and special functions},
CBMS Regional Conference Series, Vol.~21, SIAM, 1975.
%
\bibitem[64]{64}
R. Askey and M. E. H. Ismail,
{\em Recurrence relations, continued fractions, and orthogonal polynomials},
Mem. Amer. Math. Soc. 49 (1984), no. 300
%
\bibitem[79]{79}
N. M. Atakishiyev and A. U. Klimyk,
{\em On $q$-orthogonal polynomials, dual to little and big
$q$-Jacobi polynomials},
J. Math. Anal. Appl. 294 (2004), 246--257.
%
\bibitem[91]{91}
W. N. Bailey,
{\em The generating function of Jacobi polynomials},
J. London Math. Soc. 13 (1938), 8--12.
%
\bibitem[146]{146}
T. S. Chihara,
{\em An introduction to orthogonal polynomials}, Gordon and Breach, 1978;
reprinted Dover Publications, 2011.
%
\bibitem[234]{234}
G. Gasper and M. Rahman,
{\em Positivity of the Poisson kernel for the continuous
$q$-ultraspherical polynomials},
SIAM J. Math. Anal. 14 (1983), 409--420.
%
\bibitem[281]{281}
M. E. H. Ismail, J. Letessier,  G. Valent and J. Wimp,
{\em Two families of associated Wilson polynomials},
Canad. J. Math. 42 (1990), 659--695.
%
\bibitem[322]{322}
T. Koornwinder,
{\em Jacobi polynomials III. Analytic proof of the addition formula},
SIAM J. Math. Anal. 6 (1975) 533--543.
%
\bibitem[351]{351}
T. H. Koornwinder,
{\em The structure relation for Askey-Wilson polynomials},
J.~Comput.\ Appl.\ Math.\ 207 (2007), 214--226; {\tt arXiv:math/0601303v3}.
%
\bibitem[485]{485}
D. Stanton,
{\em A short proof of a generating function for Jacobi polynomials},
Proc. Amer. Math. Soc. 80 (1980), 398--400.
%
\bibitem[406]{406}
J. Meixner,
{\em Orthogonale Polynomsysteme mit einer besonderen Gestalt der erzeugenden Funktion},
J. London Math. Soc. 9 (1934), 6--13.
%
\end{thebibliography}
%
%until K6
\renewcommand{\refname}{Other references}
\begin{thebibliography}{999}
%
\bibitem[K1]{K2}
R. Askey and J. Fitch,
{\em Integral representations for Jacobi polynomials and some applications},
J. Math. Anal. Appl. 26 (1969), 411--437.
%
\bibitem[K2]{K5}
Y. Ben Cheikh and M. Gaied,
{\em Characterization of the Dunkl-classical symmetric orthogonal polynomials},
Appl. Math. Comput. 187 (2007), 105--114.
%
\bibitem[K3]{K3}
E. Feldheim,
{\em Relations entre les polynomes de Jacobi, Laguerre et Hermite},
Acta Math. 75 (1942), 117--138.
%
\bibitem[K4]{K1}
M. J. Gottlieb,
{\em Concerning some polynomials orthogonal on a finite or enumerable set of  points},
Amer. J. Math. 60 (1938), 453--458.
%
\bibitem[K5]{K4}
N. Nielsen,
{\em Recherches sur les polyn\^omes d' Hermite},
Kgl. Danske Vidensk. Selsk. Math.-Fys. Medd. I.6, K\o benhavn, 1918.
%
\bibitem[K6]{K6}
J. A. Shohat and J. D. Tamarkin,
{\em The problem of moments},
American Mathematical Society, 1943.
%
\end{thebibliography}
\end{document}